Es soll ein einfaches System zur Verwaltung von Mitarbeitern einer Firma programmiert werden.
Die Firma besitzt \qq{normale} Mitarbeiter und Manager.
Normale Mitarbeiter besitzen mindestens die Eigenschaften
\begin{itemize}
    \item Name, Gehalt, Einstellungsdatum, Personal-ID sowie
    \item eine Methode, mit der man das Gehalt um x Prozent erhöhen kann.
\end{itemize}
Manager erhalten zusätzlich zum normalen Gehalt einen Bonus für besondere Leistungen, der willkürlich einem Manager zugewiesen werden kann.
Alle Klassen sind zu kapseln.
Die Klassen Angestellter/Manager sollen in einem zu erstellenden Paket personal liegen.
Die Personalverwaltung verwendet die Klassen dieses Pakets, um Mitarbeiter anzulegen / zu entfernen, das Gehalt zu erhöhen und auszulesen.
Die Mitarbeiter sollen in einem Array gespeichert werden, dessen Werte in einer Schleife ausgelesen werden können.
Die Verwaltung soll prüfen können, ob Mitarbeiter/Manager doppelt im System vorhanden sind (doppelt = alle Eigenschaften identisch).
Für die Ausgabe der Personen soll eine Methode toString entwickelt werden.
Des weiteren soll es möglich sein, die Person mit dem höchsten und die mit dem niedrigsten Gehalt mit entsprechenden Methoden zu finden.
Verwenden Sie bei der Implementierung eine (oder mehrere) abstrakte Klasse(n) und ein generisches Array!
Erstellen Sie aus dem Paket personal ein Java-Archiv (JAR), dito für die Klasse für die Personalverwaltung.
Wie können Sie nun an der Kommandozeile Ihr Programm aufrufen?

\paragraph{Verwiesen} sei auf \qq{de.whs.ni37900.fpr.praktikum.sheet1.aufgabe2.Aufgabe2} im Anhang, diese beinhaltet eine \qq{static main(String[])}, die den Anforderungen an den Code der Aufgabe entspricht.

Um das Programm in separate .jar Dateien pro Package zu erstellen muss zunächst das Programm mittels des javac Befehls kompiliert werden.
\begin{lstlisting}[label={lst:compile-class-files}, caption={Comilieren der .java Dateien zu .class Dateien}]
javac -d out/ -classpath src/main/java/ src/main/java/de/whs/ni37900/fpr/praktikum/sheet3/aufgabe2/Main.java
\end{lstlisting}
Anschließend können die .class Dateien, mittels des jar Befehls, in .jar Archive komprimiert werden.
\begin{lstlisting}[label={lst:package-personal}, caption={Erstellen des Java Archives personal.jar}]
cd out/
jar -cf personal.jar de/whs/ni37900/fpr/praktikum/sheet3/aufgabe2/personal/*.class
\end{lstlisting}
\begin{lstlisting}[label={lst:compile-package-personalverwaltung}, caption={Erstellen des Java Archieves personalverwaltung.jar}]
cd out/
jar -cfe personalverwaltung.jar de.whs.ni37900.fpr.praktikum.sheet3.aufgabe2.Main de/whs/ni37900/fpr/praktikum/sheet3/aufgabe2/*.class
\end{lstlisting}
Zum Ausführen kann der java Befehl mit der Option cp mehrere Archive (oder kompilate) vereinen.
Bei diesem Vorgehen kann der Einstiegspunkt nicht aus den Metadaten der Archive extrahiert werden, da potenziell mehrere angegeben seien können.
Deshalb muss dieser manuell angegeben werden.
\begin{lstlisting}[label={lst:compile-run-personalverwaltung}, caption={Starten der Personalverwaltung}]
cd out/
java -cp personal.jar:personalverwaltung.jar de.whs.ni37900.fpr.praktikum.sheet3.aufgabe2.Main
\end{lstlisting}
