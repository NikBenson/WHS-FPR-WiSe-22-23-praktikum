Es gebe folgende Klassen:
\begin{lstlisting}[label={lst:superclass-original}, caption={Superclass}, language=java]
public class Superclass {
    protected static void meineMethode() {
        return;
    }
}
\end{lstlisting}
\begin{lstlisting}[label={lst:subclass-original}, caption={Subclass}, language=java]
public class Subclass extends Superclass {
    public static void main(String[] args) {
        meineMethode();
    }
}
\end{lstlisting}

\begin{enumerate}
    \item Warum ist meineMethode\(\) in \qq{Subclass} sichtbar?
    \item Was können sie tun, um zu erreichen, dass diese Methode dort unsichtbar ist, ohne der Klasse oder der Methode einen anderen Zugriffsmodifikator zu geben?
\end{enumerate}

\paragraph{Der} implizite Zugriffsmodifikator, wenn nicht manuell angegeben, ist \qq{package}.
Das heißt, dass diem Methode nur im selben package sichtbar ist.
Wollen wir nun verhindern, dass meineMethode\(\) sichtbar ist, so lagern wir eine der beiden Klassen in ein anderes Package aus.